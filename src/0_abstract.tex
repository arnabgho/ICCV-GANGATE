\begin{abstract}
We propose an interactive GAN-based sketch-to-image translation method 
%We propose a new method involving an interactive GAN-based recommender
that helps novice users easily create images of simple objects.
The user starts with a sparse sketch and a desired object category, and the network then recommends its plausible completion and shows a corresponding synthesized image. This enables a feedback loop, where the user can edit the sketch based on the network's recommendations, while the network is able to better synthesize the image that the user might have in mind. 
%synthesizing the object with the desired shape they may have in mind.
In order to use a single model for a wide array of object classes, we introduce a gating-based approach for class conditioning, which allows us to generate distinct classes without feature mixing, from a single generator network.

%Effectively incorporating low-dimensional conditioning is a crucial, yet relatively unexplored, aspect of multiclass, multimodal, and multitask image-to-image translation.
%We systematically investigate variants of injecting conditioning into GAN architectures. We propose a soft-gating mechanism, which learns to ``select" which parts of a ResNet~\cite{he2016deep} are used, based on the conditioner.
%We validate on a challenging outline-to-image task, mapping from a sparse sketch with no internal structure. 
%Class conditioning is especially important, and the soft-gating mechanism enables plausible generations where naive concatenation fails.
%The method is also effective on standard sketch-to-image and day-to-night tasks.
%Additionally, the gating better fulfills the InfoGAN~\cite{chen2016infogan} objective, maximizing mutual information between the output and the conditioner,
% in an effective way,
%enabling diverse generations for image-to-image translation. 
% We demonstrate that our approach is able to generate high quality multiclass image translations, outperforming prior state-of-the-art methods.

%We propose a method that allows us to generate images belonging to multiple domains using a single network.
%Our approach is based on a GAN framework with two separate branches, a fully residual generator and discriminator, and a separate smaller gating network.
%The gating network selects residual blocks from the generator and discriminator based on some conditioning.
%We show that such an approach is able to produce high quality multi-class image generation, both in an class-conditioned image-to-image translation task, as well as in an unsupervised image generation task where diversity is learned.
%This method allows for significantly smaller model sizes than previous multi-class approaches by taking advantage of similarities across classes. 
%We analyze our gating network to show that it leads to subnetworks based on the residual blocks that are active for a particular class.
%This approach as well as helps inject low dimensional information into a network more effectively than just concatenating channel-wise after replication to match the image dimension. 
%Information theoretic results also show it to be a much stronger form of conditioning than naive concatenation. 
%We apply our approach on a novel setting of multi-class outline-to-image generation, where baseline solutions fail to generate good results, while our model successfully tackles the multi-class image generation setting. 
\end{abstract}